\documentclass{sajzk}

\begin{document}
\section{Geometrische Algebra}
\label{f35d}
\lhead{:mathe:ga:}

Geometrische Algebra $\mathscr{G}$ ist eine Erweiterung eines Vektorraumes $V$
mit Skalarprodukt. Die Elemente representieren geometrische Objekte aller
Dimensionen des Vektorraumes. Algebraische Operationen auf $\mathscr{G}$
manipuieren diese Objekte.

\begin{itemize}
    \item 0-Dimensionale Objekte: Punkte, skalare Zahlen
    \item 1-Dimensionale Objekte: Geraden, Vektoren
    \item 2-Dimensionale Objekte: Flächen, Bivektoren
    \item 3-Dimensionale Objekte: Räume, Trivektoren.
    \item ...
\end{itemize}

Durch das geometrische Produkt zweier Vektoren könnne orientierte Flächen
entstehen. Weiterhin können durch das geometrische Produkt dreier Vektoren
orientierte Räume entstehen.

\subsection{Referenzen}
\begin{itemize}
    \item \href{3g4f.pdf}{Reeller Vektorraum} 3g4f.tex
    \item \href{16ea.pdf}{Grassmanns Idee} 16ea.tex
    \item \href{81js.pdf}{Geometrisches Produkt} 81js.tex
    \item \href{il6v.pdf}{Axiomatische sichtweise} il6v.tex
    \item \href{e6nk.pdf}{Basisvektoren für GA in 2D} e6nk.tex
    \item \href{fw8i.pdf}{Basisvektoren für GA in 3D} fw8i.tex
    \item \href{fmjb.pdf}{Projektion und Rejektion} fmjb.tex
    \item \href{kikd.pdf}{$k$-Blades} kikd.tex
    \item \href{93t3.pdf}{$k$-Vektoren} 93t3.tex
    \item \href{d1fv.pdf}{Ein Multivektor} d1fv.tex
    \item \href{eelx.pdf}{Die Umkehrung eines Multivektors} eelx.tex
    \item \href{oagu.pdf}{Der $k$-Vektorteil eines Multivektors} oagu.tex
    \item \href{deuk.pdf}{Die Norm eines Multivektors} deuk.tex
\end{itemize}
- [Das innere und äußere Produkt](bzmt.md)

\subsection{Literatur}
\begin{itemize}
    \item Alan Macdonald - Linear and Geometric Algebra (2021) Seite 73
    \item David Hestenes, Garret Sobczyk - Clifford Algebra to Geometric Calculus A Unified Language for Mathematics and Physics (1984)
\end{itemize}
\end{document}
