\documentclass{sajzk}

\begin{document}
\section{Disjunktor-Einführungsregel \texorpdfstring{$(\lor E)$}{(or E)}}
\label{7yg4}
\lhead{:mathe:logik:kalkül:}
Enthält eine Ableitung die Formel A, dann darf die Disjunktion der Formel A mit
einer beliebigen Formel B in eine neue Zeile aufgenommen werden. Die neue
Annahmeliste ist eine Kopie der Annahmeliste der Oberformel.

\begin{center}
  \begin{minipage}[t]{0.4\textwidth}
    \begin{tabular}{|c|c|c|c|}
      \hline
      Ann.               & Nr.    & Formel       & Regel \\
      \hline
      $\alpha$           & (k)    & A            & AE \\
      \hline
      $\alpha$           & (m)    & A $\lor$ B  & k $\lor$E \\
      \hline
    \end{tabular}
  \end{minipage}
  \begin{minipage}[t]{0.4\textwidth}
    \begin{tabular}{|c|c|c|c|}
      \hline
      Ann.               & Nr.    & Formel       & Regel \\
      \hline
      $\alpha$           & (k)    & A            & AE \\
      \hline
      $\alpha$           & (m)    & B $\lor$ A  & k $\lor$E \\
      \hline
    \end{tabular}
  \end{minipage}
\end{center}

\textsc{Beispiele:}
\begin{center}
    \[P \therefore P \lor Q\] \\
\begin{tabular}{|c|c|c|c|}
  \hline
  Annahme            & Nummer & Formel       & Regel \\
  \hline
  1                  & (1)    & $P$          & AE \\
  \hline
  1                  & (2)    & $P \lor Q$   & 1 $\lor$E \\
  \hline
\end{tabular}
\end{center}
Es gilt: $P \vdash_J P \lor Q$
\newpage
\begin{center}
    \[(P \lor Q) \ra R \therefore P \ra  R\]
\begin{tabular}{|c|c|c|c|}
  \hline
  Annahme            & Nummer & Formel                 & Regel \\
  \hline
  1                  & (1)    & $(P \lor Q) \ra R$     & AE \\
  \hline
  $2^\star$          & (2)    & $P$                    & AE \\
  \hline
  $2^\star$          & (3)    & $P \lor Q$             & 2 $\lor$E \\
  \hline
  $1, 2^\star$       & (4)    & $R$                    & 1,3 MPP \\
  \hline
  1                  & (5)    & $P \ra R$              & 2,4 K \\
  \hline
\end{tabular}
\end{center}
Es gilt: $(P \lor Q) \ra R \vdash_J P \ra  R$
\subsection{Referenzen}
\begin{itemize}
    \item \href{8hkr.pdf}{Kalkül der Aussagenlogik} 8hkr.tex
\end{itemize}

\subsection{Literatur}
\begin{itemize}
    \item Timm Lampert - Klassische Logik (2004) Seite 101-102
\end{itemize}
\end{document}
