\documentclass{sajzk}

\begin{document}
\section{Konjunktor-Beseitigungsregel \texorpdfstring{($\land$B)}{(and B)}}
\label{4k16}
\lhead{:mathe:logik:kalkül:}
Enthält die Ableitung ein Konjuntkor der Formeln A und B, dann darf eine der
beiden Glieder (A oder B) in eine neue Zeile aufgenommen werden. Die neue
Annahmeliste ist eine Kopie der Oberformel.

\begin{center}
  \begin{minipage}[t]{0.4\textwidth}
    \begin{tabular}{|c|c|c|c|}
      \hline
      Ann.               & Nr.    & Formel       & Regel \\
      \hline
      $\alpha$           & (k)    & A $\land$ B  & AE \\
      \hline
      $\alpha$           & (l)    & A            & k $\land$B \\
      \hline
    \end{tabular}
  \end{minipage}
  \begin{minipage}[t]{0.4\textwidth}
    \begin{tabular}{|c|c|c|c|}
      \hline
      Ann.               & Nr.    & Formel       & Regel \\
      \hline
      $\alpha$           & (k)    & A $\land$ B  & AE \\
      \hline
      $\alpha$           & (l)    & B            & k $\land$B \\
      \hline
    \end{tabular}
  \end{minipage}
\end{center}

\textsc{Beispiele:}
\begin{center}
    \[P \land Q \therefore Q \land P\] \\
\begin{tabular}{|c|c|c|c|}
  \hline
  Annahme            & Nummer & Formel       & Regel \\
  \hline
  1                  & (1)    & $P \land Q$  & AE \\
  \hline
  1                  & (2)    & $P$          & 1 $\land$B \\
  \hline
  1                  & (3)    & $Q$          & 1 $\land$B \\
  \hline
  1                  & (4)    & $Q \land P$  & 3,2 $\land$E \\
  \hline
\end{tabular}
\end{center}
Es gilt: $P \land Q \vdash_J Q \land P$
\newpage
\begin{center}
    \[P \ra Q, P \ra R \therefore P \ra (Q\land R)\]
\begin{tabular}{|c|c|c|c|}
  \hline
  Annahme            & Nummer & Formel                 & Regel \\
  \hline
  1                  & (1)    & $P \ra Q$              & AE \\
  \hline
  2                  & (2)    & $P \ra R$              & AE \\
  \hline
  $3^\star$          & (3)    & $P$                    & AE \\
  \hline
  $1, 3^\star$       & (4)    & $Q$                    & 1,3 MPP \\
  \hline
  $2, 3^\star$       & (5)    & $R$                    & 2,3 MPP \\
  \hline
  $1, 2, 3^\star$    & (6)    & $Q \land R$            & 4,5 $\land$E \\
  \hline
  $1, 2$             & (7)    & $P \ra (Q \land R)$    & 3,6 K \\
  \hline
\end{tabular}
\end{center}
Es gilt: $P \ra Q, P \ra R \vdash_J P \ra (Q\land R)$
\subsection{Referenzen}
\begin{itemize}
    \item \href{8hkr.pdf}{Kalkül der Aussagenlogik} 8hkr.tex
\end{itemize}

\subsection{Literatur}
\begin{itemize}
    \item Timm Lampert - Klassische Logik (2004) Seite 100-101
\end{itemize}
\end{document}
