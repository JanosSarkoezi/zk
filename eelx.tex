\documentclass{sajzk}

\begin{document}
\section{Die Umkehrung eines Multivektors}
\label{eelx}
\lhead{:mathe:ga:}

\begin{itemize}
  \item  Definition nach Macdonald: Sei $A=a_1\cdots a_k$ ein $k$-Blade. Dann ist die
  Umkehrung von $A$ als
  \[
  A^\dagger = a_k\cdots a_1
  \]
  definiert. Für einen Multivektor $A$ wende die obige Gleichung auf die
  $k$-Vektorteile an.

  \item Definition nach Hestenes: Seien $A$ und $B$ Multivektoren. Dann wird die
  Umkehrung defniert als
  \begin{align*}
      (AB)^\dagger &= B^\dagger A^\dagger \\
      (A+B)^\dagger &= A^\dagger + B^\dagger \\
      \langle A^\dagger \rangle_0 &= \langle A \rangle_0\\
      a^\dagger &= a\hspace{3ex}\textrm{wenn}\hspace{1ex} a= \langle a\rangle_1
  \end{align*}
  Dabei ist mit $\langle A \rangle_0$ der $0$-Vektorteil des Multivektors $A$
  gemeint. Also der skalare Teil des Multivektors $A$.
\end{itemize}

\subsection{Referenzen}
\begin{itemize}
    \item \href{f35d.pdf}{Geometrische Algebra} f35d.tex
    \item \href{d1fv.pdf}{Ein Multivektor} d1fv.tex
    \item \href{oagu.pdf}{Der $k$-Vektorteil eines Multivektors} oagu.tex
\end{itemize}

\subsection{Literatur}
\begin{itemize}
    \item Alan Macdonald - Linear and Geometric Algebra (2021) Seite 97
    \item David Hestenes - New Foundations for Classical Mechanics (2002) Seite 45
\end{itemize}
\end{document}
