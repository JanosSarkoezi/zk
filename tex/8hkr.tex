\documentclass{sajzk}

\begin{document}
\section{Kalkül der Aussagenlogik} 
\label{8hkr}
\lhead{:mathe:logik:kalkül:}

Ein Kalkül ist eine Menge von Regeln für die zulässige Umformung von Formeln.

\subsection{Referenzen} 

\begin{itemize}
  \item \href{3u33.pdf}{Ableitungen} 3u33.tex
  \item \href{yyix.pdf}{Annahmen Einführung (AE)} yyix.tex
  \item \href{69fy.pdf}{Modus Ponendo Ponens (MPP)} 69fy.tex
  \item \href{30ol.pdf}{Modus Tollendo Tollens (MTT)} 30ol.tex
  \item \href{5qwp.pdf}{Konditionlaisierung (K)} 5qwp.tex
  \item \href{zzno.pdf}{Konjunktor-Einführungsregel (and E)} zzno.tex
  \item \href{4k16.pdf}{Konjunktor-Beseitigungsregel (and B)} 4k16.tex
  \item \href{7yg4.pdf}{Disjunktor-Einführungsregel (or E)} 7yg4.tex
  \item \href{4sni.pdf}{Disjunktor-Beseitigungsregel (or B)} 4sni.tex
  \item \href{o9ls.pdf}{Reductio Ad Absurdum (RAA)} o9ls.tex
  \item \href{is2y.pdf}{Doppelte Negation (DNE, DNB)} is2y.tex
\end{itemize}

\subsection{Literatur} 

\begin{itemize}
  \item Timm Lampert - Klassische Logik (2004)
\end{itemize}
\end{document}
