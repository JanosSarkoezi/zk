\documentclass[a4paper,12pt]{article}
\usepackage[utf8]{inputenc}
\usepackage{amsmath, amssymb}
\usepackage{geometry}
\geometry{margin=2.5cm}
\usepackage{hyperref}
\title{Ableitung der funktionaldeterminante und bewegung in einem zeitabhängigen kraftfeld}
\author{erstellt mit hilfe von chatgpt}
\date{\today}

\begin{document}

\maketitle

\section*{1. gradient eines konstanten gravitationsfeldes}

ein konstantes gravitationsfeld nahe der erdoberfläche hat die form:
\[
\vec{g} = (0, 0, -g)
\]
der gradient eines vektorfeldes ist ein tensor:
\[
\nabla \vec{g} = \begin{pmatrix}
\frac{\partial g_x}{\partial x} & \frac{\partial g_x}{\partial y} & \frac{\partial g_x}{\partial z} \\
\frac{\partial g_y}{\partial x} & \frac{\partial g_y}{\partial y} & \frac{\partial g_y}{\partial z} \\
\frac{\partial g_z}{\partial x} & \frac{\partial g_z}{\partial y} & \frac{\partial g_z}{\partial z}
\end{pmatrix}
\]
da $\vec{g}$ konstant ist, sind alle ableitungen null:
\[
\nabla \vec{g} = \mathbf{0}
\]

\section*{2. zentralfeld und gradient}

in einem zentralen gravitationsfeld (z. b. $\vec{g} = -g m \frac{\vec{r}}{r^3}$) ist der gradient im allgemeinen nicht null, da das feld vom ort abhängt. dort gilt:
\[
\nabla \vec{g} \neq 0
\]

\section*{3. gradient eines zeitabhängigen vektorfeldes}

sei $\vec{f}(x, y, z, t) = (0, 0, a t)$. dann ist der gradient:
\[
\nabla \vec{f} = \begin{pmatrix}
0 & 0 & 0 \\
0 & 0 & 0 \\
0 & 0 & 0
\end{pmatrix}
\]
da $f_z$ nur von $t$ abhängt, sind alle räumlichen ableitungen null.

\section*{4. bewegung eines punktkörpers in einem solchen feld}

die bewegungsgleichung nach newton lautet:
\[
m \ddot{z} = a t \rightarrow \ddot{z} = \frac{a}{m} t
\]
integration ergibt:
\[
\dot{z}(t) = \frac{a}{2m} t^2 + v_0, \quad z(t) = \frac{a}{6m} t^3 + v_0 t + z_0
\]

\section*{5. beschreibung mit kontinuumsmechanik}

die bewegung kann durch die cauchy-gleichung beschrieben werden:
\[
\rho \frac{d\vec{v}}{dt} = \nabla \cdot \boldsymbol{\sigma} + \rho \vec{b}
\]
für einen punktförmigen körper ohne spannung:
\[
\frac{d\vec{v}}{dt} = \vec{b} = (0, 0, a t)
\]
ergebnis wie zuvor:
\[
z(t) = \frac{a}{6} t^3 + v_0 t + z_0
\]

\section*{6. ableitung der funktionaldeterminante}

die funktionaldeterminante ist:
\[
j(t) = \det\left( \frac{\partial \vec{\chi}}{\partial \vec{x}} \right)
\]
die ableitung ergibt:
\[
\dot{j}(t) = j(t) \cdot \text{tr}\left( \mathbf{f}^{-1}(t) \frac{d \mathbf{f}(t)}{dt} \right)
\]
in der kontinuumsmechanik:
\[
\dot{j}(t) = j(t) \cdot \nabla \cdot \vec{v}
\]

\section*{7. herleitung dieser formel}

für $j(t) = \det(\mathbf{f}(t))$, mit $\mathbf{f} = \frac{\partial \vec{\chi}}{\partial \vec{x}}$ gilt:
\[
\frac{d}{dt} \det(\mathbf{f}(t)) = \det(\mathbf{f}(t)) \cdot \text{tr}\left( \mathbf{f}^{-1}(t) \frac{d\mathbf{f}(t)}{dt} \right)
\]
dies basiert auf der matrixregel für die ableitung der determinante:
\[
\frac{d}{dt} \det(\mathbf{a}(t)) = \det(\mathbf{a}(t)) \cdot \text{tr}\left( \mathbf{a}^{-1}(t) \frac{d\mathbf{a}(t)}{dt} \right)
\]

\end{document}
