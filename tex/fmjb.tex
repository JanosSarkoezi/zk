\documentclass{sajzk}

\begin{document}
\section{Projektion und Rejektion}
\label{fmjb}
\lhead{:mathe:ga:}

Eine Porjektion wird in einem Vektorraum $V$ mit Skalarprodukt vorgenommen.
Dabei wird immer ein Vektor $v$ auf einen Unterraum $U$ des Vektorraumes $V$
projiziert.

Die Rejektion eines Vektors $v$ ist ein Elemment des orthogonalen Komplements
$U^\perp$ des Unterraumes $U$ der Projektion.

Mit der Projektion und Rejektion wird der Vektor $v$ in zwei Komponenten
aufgeteilt. Die erste Komponente wird mit $v_\parallel$ und die zweite
Komponente mit $v_\perp$ bezeichnet.
$$v = v_\parallel + v_\perp\hspace{3ex} v_\parallel \in U, v_\perp \in U^\perp$$
Diese Darstellung kann nur auf eine Weise vorgenommen werden. Dies hängt mit
der Aussage 
$$U\cap U^\perp = \{0\}$$
zusammen.

\subsection{Referenzen}
\begin{itemize}
    \item \href{f35d.pdf}{Geometrische Algebra} f35d.tex
    \item \href{81js.pdf}{Geometrisches Produkt} 81js.tex
\end{itemize}
- [Projektion auf einen 1D Unterraum](dog5.md)

\subsection{Literatur}
\begin{itemize}
    \item Alan Macdonald - Linear and Geometric Algebra (2021) Seite 52
\end{itemize}
\end{document}
