\documentclass{sajzk}

\begin{document}
\section{Gesetz der doppelten Negation (GDN)}
\label{hubq}
\lhead{:mathe:logik:kalkül:}

\textbf{GDN1:} $\vdash A\ra\lnot\lnot A$

\textbf{Theorem} $\vdash P\ra\lnot\lnot P$

\textbf{Ableitung:}
\begin{center}
\begin{tabular}{|c|c|c|c|}
  \hline
  Annahme            & Nr. & Formel                     & Regel \\
  \hline
  $1^\star$          & (1)    & $P$                     & AE \\
  \hline
  $1^\star$          & (2)    & $\lnot\lnot P$          & 1 DNE \\
  \hline
                     & (3)    & $P\ra\lnot\lnot P$      & \\
  \hline
\end{tabular}
\end{center}
\newpage
\textbf{GDN2:} $\vdash \lnot\lnot A\ra A$

\textbf{Theorem} $\vdash \lnot\lnot P\ra P$

\textbf{Ableitung:}
\begin{center}
\begin{tabular}{|c|c|c|c|}
  \hline
  Annahme            & Nr. & Formel                     & Regel \\
  \hline
  $1^\star$          & (1)    & $\lnot\lnot P$          & AE \\
  \hline
  $1^\star$          & (2)    & $P$                     & 1 DNB \\
  \hline
                     & (3)    & $\lnot\lnot P\ra P$     & \\
  \hline
\end{tabular}
\end{center}
\subsection{Referenzen}
\begin{itemize}
  \item \href{s52i.pdf}{Logische Gesetze} s52i.tex
\end{itemize}

\subsection{Literatur}
\begin{itemize}
    \item Timm Lampert - Klassische Logik (2004) Seite 121-122
\end{itemize}
\end{document}
