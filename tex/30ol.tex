\documentclass{sajzk}

\begin{document}
\section{Modus Tollendo Tollens (MTT)}
\label{30ol}
\lhead{:mathe:logik:kalkül:}

Enthält eine Ableitung sowohl eine Formel mit $'\ra'$ als Hauptjunktor
als auch die Negation der Konsequenz dieser Formel, dann darf die Negation der
Antezedenz in eine \textit{neue} Zeile aufgenommen werden. Die Annahmeliste der
neuen Zeile wird zusammengesetzt aus den Annahmelisten der beiden Oberformeln.

\begin{center}
\begin{tabular}{|c|c|c|c|}
  \hline
  Annahme         & Nummer & Formel        & Regel \\
  \hline
  $\alpha$        & (k)    & $\lnot$B      &  \\
  \hline
  $\beta$         & (l)    & A $\ra$ B     &  \\
  \hline
  $\alpha, \beta$ & (m)    & $\lnot$A      & l,k MTT \\
  \hline
\end{tabular}
\end{center}

Die Reihenfolge des Vorkommens der Oberformeln ist beliebig.

\textsc{Beispiele:}
\begin{center}
    \[\lnot Q, P \ra Q \therefore \lnot P\] \\

\begin{tabular}{|c|c|c|c|}
  \hline
  Annahme         & Nummer  & Formel & Regel \\
  \hline
  1   & (1)    & $\lnot Q$  & AE \\
  \hline
  2   & (2)    & $P \ra Q$  & AE \\
  \hline
  1,2 & (3)    & $\lnot P$  & 2,1 MTT \\
  \hline
\end{tabular}
\end{center}

Es gilt: $\lnot Q, P \ra Q \vdash_J \lnot P$
\newpage
\begin{center}
    \[P \ra (Q\ra R), P ,\lnot R \therefore \lnot Q\] \\

\begin{tabular}{|c|c|c|c|}
  \hline
  Annahme        & Nummer & Formel   & Regel \\
  \hline
  1     & (1)    & $P \ra (Q\ra R)$  & AE \\
  \hline
  2     & (2)    & $P$               & AE \\
  \hline
  3     & (3)    & $\lnot R$         & AE \\
  \hline
  1,2   & (4)    & $Q \ra R$         & 1,2 MPP \\
  \hline
  1,2,3 & (5)    & $\lnot Q$         & 3,4 MTT \\
  \hline
\end{tabular}
\end{center}

Es gilt: $P \ra (Q\ra R), P, \lnot R \vdash_J \lnot Q$
\subsection{Referenzen}
\begin{itemize}
    \item \href{8hkr.pdf}{Kalkül der Aussagenlogik} 8hkr.tex
\end{itemize}

\subsection{Literatur}
\begin{itemize}
    \item Timm Lampert - Klassische Logik (2004) Seite 95-96
\end{itemize}
\end{document}
