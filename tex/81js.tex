\documentclass{sajzk}

\begin{document}
\section{Geometrisches Produkt}
\label{81js}
\lhead{:mathe:ga:}

Das geometrisches Produkt kann mit der Einführung eines Dachprodukts und des 
Skalarprodukts definiert werden. Das Dachprodukt wird auch als äußeres
Produkt bezeichnet.

Seien zwei Vektoren $u$ und $v$ aus einem n-dimensionalen reellen Vektorraum mit
Skalarprodukt gegeben, dann ist das geometrisches Produkt definiert als:

$$
uv = u \cdot v + u \wedge v
$$

Dabei bezeichnen die Symbole $\cdot$ und $\wedge$ das Skalarprodukt und das
äußere Produkt.

\subsection{Referenzen}

\begin{itemize}
    \item \href{3g4f.pdf}{Reeller Vektorraum} 3g4f.tex
    \item \href{fuw3.pdf}{Äußeres Produkt} fuw3.tex
    \item \href{f35d.pdf}{Geometrische Algebra} f35d.tex
    \item \href{il6v.pdf}{Axiomatische sichtweise} il6v.tex
\end{itemize}

- [Eigenschaften des geometrischen Produktes](97tw.md)

\subsection{Literatur}
\begin{itemize}
    \item Alan Macdonald - Linear and Geometric Algebra (2021) Seite 82
\end{itemize}
\end{document}
