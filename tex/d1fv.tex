\documentclass{sajzk}

\begin{document}
\section{Ein Multivektor}
\label{d1fv}
\lhead{:mathe:ga:}
Ein Multivektor ist ein Element aus dem $\mathscr{G}^n$.

\textsc{Beispiele:}
\begin{itemize}
  \item Ein Multivektor aus $\mathscr{G}^3$ ist somit darstellbar als die
  Linearkombination der Basisvektoren aus diesem Raum. In der allgemeinsten From
  ist somit ein Multivektor aus $\mathscr{G}^3$
  \[
  \alpha_0 + \alpha_1 e_1 + \alpha_2 e_2 + \alpha_3 e_3 + \alpha_4 e_1e_2 + \alpha_5 e_1e_3 + \alpha_6 e_2e_3 + \alpha_7 e_1e_2e_3
  \]
  mit $\alpha_0, ...,\alpha_7\in\R$.
  \item Ein Multivektor aus $\mathscr{G}^2$ ist somit
  \[
  \alpha_0 + \alpha_1 e_1 + \alpha_2 e_2 + \alpha_3 e_1e_2
  \]
  mit $\alpha_0, ...,\alpha_3\in\R$.
\end{itemize}

\subsection{Referenzen}
\begin{itemize}
    \item \href{f35d.pdf}{Geometrische Algebra} f35d.tex
    \item \href{e6nk.pdf}{Basisvektoren für GA in 2D} e6nk.tex
    \item \href{fw8i.pdf}{Basisvektoren für GA in 3D} fw8i.tex
    \item \href{f35d.pdf}{RGeometrische Algebra} f35d.tex
\end{itemize}
- [Der $k$-Vektorteil eines Multivektors](oagu.md)

\subsection{Literatur}
\begin{itemize}
    \item Alan Macdonald - Linear and Geometric Algebra (2021) Seite 82
    \item Alan Macdonald - Linear and Geometric Algebra (2021) Seite 93
    \item David Hestenes - New Foundations for Classical Mechanics (2002) Seite 34
\end{itemize}
\end{document}
