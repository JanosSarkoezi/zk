\documentclass{sajzk}

\begin{document}
\section{Konditionlaisierung (K)}
\label{5qwp}
\lhead{:mathe:logik:kalkül:}
Ist A einer der \textit{Annahmen}, von denen die Ablietung von B abhängt, so
darf das Konditional mit A als Antezedenz und B als Konsequenz in eine neue
Zeile aufgenommen werden. Die neue Annahmeliste ergibt sich aus der Annahmeliste
von B durch Streichung der Zeilennummer der Annahme A.

\begin{center}
\begin{tabular}{|c|c|c|c|}
  \hline
  Annahme            & Nummer & Formel     & Regel \\
  \hline
  $k^\star$          & (k)    & A          & AE \\
  \hline
  $\alpha, k^\star$  & (l)    & B          &  \\
  \hline
  $\alpha$           & (m)    & A $\ra$ B  & k,l K \\
  \hline
\end{tabular}
\end{center}
Das Sternchen hinter der Zeilennummer der Formel A, kennzeichnet diese als
Hilfsannahme für die Ableitung der Formel in der Zeile m, in deren Annahmeliste
nicht mehr auf die Formel A bezug genommen wird.

\textsc{Beispiele:}
\begin{center}
    \[P \ra Q \therefore \lnot Q \ra \lnot P\] \\

\begin{tabular}{|c|c|c|c|}
  \hline
  Annahme     & Nummer & Formel               & Regel \\
  \hline
  1           & (1)    & $P \ra Q$            & AE \\
  \hline
  $2^\star$   & (2)    & $\lnot Q$            & AE \\
  \hline
  $1,2^\star$ & (3)    & $\lnot P$            & 1,2 MTT \\
  \hline
  1           & (4)   & $\lnot Q\ra \lnot P$  & 2,3 K \\
  \hline
\end{tabular}
\end{center}
Es gilt: $P \ra Q \vdash_J \lnot Q \ra \lnot P$

\begin{center}
    \[P \ra (Q \ra R) \therefore  Q \ra (P \ra R)\] \\

\begin{tabular}{|c|c|c|c|}
  \hline
  Annahme             & Nummer & Formel              & Regel \\
  \hline
  1                   & (1)    & $P \ra (Q \ra R)$   & AE \\
  \hline
  $2^\star$           & (2)    & $P$                 & AE \\
  \hline
  $3^\star$           & (3)    & $Q$                 & AE \\
  \hline
  $1,2^\star$         & (4)    & $Q \ra R$           & 1,2 MPP \\
  \hline
  $1,2^\star,3^\star$ & (5)    & $R$                 & 4,3 MPP \\
  \hline
  $1,3^\star$         & (6)    & $P\ra R$            & 2,5 K \\
  \hline
  $1$                 & (7)    & $Q \ra (P\ra R)$    & 3,5 K \\
  \hline
\end{tabular}
\end{center}
Es gilt: $P \ra (Q \ra R) \vdash_J Q \ra (P \ra R)$

\begin{center}
    \[P \ra P\] \\

\begin{tabular}{|c|c|c|c|}
  \hline
  Annahme         & Nummer  & Formel      & Regel \\
  \hline
  $1^\star$       & (1)     & $P$         & AE \\
  \hline
                  & (2)     & $P \ra P$   & 1,1 K \\
  \hline
\end{tabular}
\end{center}
Es gilt: $\vdash_J P \ra P$
\subsection{Referenzen}
\begin{itemize}
    \item \href{8hkr.pdf}{Kalkül der Aussagenlogik} 8hkr.tex
\end{itemize}

\subsection{Literatur}
\begin{itemize}
    \item Timm Lampert - Klassische Logik (2004) Seite 96-97
\end{itemize}
\end{document}
