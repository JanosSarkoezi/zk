\documentclass{sajzk}

\begin{document}
\section{\texorpdfstring{$k$-Vektoren}{k-Vektoren}}
\label{93t3}
\lhead{:mathe:ga:}
Die $k$-Vektoren können auf verschiedene Arten Definiert werden:

\begin{itemize}
    \item \href{yher.pdf}{$k$-Vektoren nach Macdonalds} yher.tex
    \item \href{2b2x.pdf}{$k$-Vektoren nach Hestenes} 2b2x.tex
    \item \href{ucnm.pdf}{$k$-Vektoren nach Chisolm} ucnm.tex
\end{itemize}

\subsection{Referenzen}
\begin{itemize}
    \item \href{f35d.pdf}{Geometrische Algebra} f35d.tex
    \item \href{81js.pdf}{Geometrisches Produkt} 81js.tex
    \item \href{e6nk.pdf}{Basisvektoren für GA in 2D} e6nk.tex
    \item \href{kikd.pdf}{$k$-Blades} kikd.tex
    \item \href{oagu.pdf}{Der $k$-Vektorteil eines Multivektors} oagu.tex
\end{itemize}

\subsection{Literatur}
\begin{itemize}
  \item Alan Macdonald - Linear and Geometric Algebra (2021) Seite 82
  \item David Hestenes - New Foundations for Classical Mechanics (2002) Seite 41
  \item Eric Chisolm - Geomeric Algebra (2012) Seite 12
\end{itemize}
\end{document}
