\documentclass{sajzk}

\begin{document}
\section{Reeller Vektorraum} 
\label{3g4f}
\lhead{:mathe:la:}


Ein reller Vektorraum $(V, +, \cdot)$ besteht aus einer Menge $V$ und zwei
Operationen auf dieser Menge. Eine Addition $+$ und eine skalare
Multiplikation $\cdot$ mit einer rellen Zahl.

Bezüglich der Addition $+$ soll diese Menge $V$ eine abelsche Gruppe bilden.

\begin{enumerate}
    \item $x + y = y + x$ Kommutativität.
    \item $x + (y + z) = (x + y) + z$ Assoziativität.
    \item $x + 0 = x$ Es gibt ein neutrales Element $0$.
    \item $x + y = 0$ Es gibt ein inverses Element $y$. Wird als $-x$ bezeichnet.
\end{enumerate}

Skalare Multiplikation mit $\alpha, \beta \in \R$ und $x, y \in V$ soll
folgenden Regeln genügen:

\begin{enumerate}
    \setcounter{enumi}{4}
    \item $\alpha\cdot(x + y) = \alpha\cdot x +\alpha\cdot y$ Distributivität.
    \item $(\alpha + \beta)\cdot x = \alpha\cdot x + \beta\cdot x$
    \item $(\alpha\cdot\beta)\cdot x = \alpha\cdot(\beta\cdot x)$
    \item $1\cdot x = x$ Neutralität des Einselements 1 der reellen Zahlen.
\end{enumerate}

Werden diese Axiome erfüllt, so wird diese Menge $V$ als ein reeller
Vektorraum bezeichnet.

\subsection{Referenzen}
\begin{itemize}
  \item \href{jtfm.pdf}{Lineare Algebra} jtfm.tex
  \item \href{f35d.pdf}{Geometrische Algebra} f35d.tex
\end{itemize}

\subsection{Literatur}
\begin{itemize}
  \item Alan Macdonald - Linear and Geometric Algebra (2021) Seite 15
\end{itemize}
\end{document}
