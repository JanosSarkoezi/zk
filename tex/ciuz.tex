\documentclass{sajzk}

\begin{document}
\section{Axiome von Hestenes}
\label{ciuz}
\lhead{:mathe:ga:}

Die Elemente der Geometrischen Algebra $\mathscr{G}$ heißen Multivektoren. Die
Summe und das Produkt zwischen den Multivektoren $A, B, C, ...$ haben folgende
Eigenschaften:

Kommutativität der Addition
\begin{itemize}
    \item $A + B = B + A$
\end{itemize}

Assoziativität der Addition und Multiplikation
\begin{itemize}
    \item $(A + B) + C = A + (B + C)$
    \item $(AB)C = A(BC)$
\end{itemize}

Distributivität
\begin{itemize}
    \item $A(B + C) = AB + AC$
    \item $(B + C)A = BA + CA$
\end{itemize}

Existenz der additiven und multiplikativen Einheiten $0, 1$
\begin{itemize}
    \item $A + 0 = A$
    \item $1A = A$
\end{itemize}

Existenz der additiven Inversen
\begin{itemize}
    \item $A + (-A) = 0$
\end{itemize}

Kommutativität der skalaren Multiplikation $\lambda$
\begin{itemize}
    \item $\lambda A = A \lambda$
\end{itemize}

Eigenschat eines Vektors $a$ ungleich null. Das Quadrat eines Vektors $a$ ist
die positive Zahl $|a|^2$ (Die Länge zum Quadrat des Vektors $a$).

\begin{itemize}
    \item $aa = a^2 := |a|^2 > 0$
\end{itemize}

\subsection{Referencen}
\begin{itemize}
    \item \href{f35d.pdf}{Geometrische Algebra} f35d.tex
    \item \href{il6v.pdf}{Axiomatische sichtweise} il6v.tex
\end{itemize}

\subsection{Literatur}
\begin{itemize}
    \item David Hestenes - New Foundations for Classical Mechanics (2002) Seite 35
    \item avid Hestenes, Garret Sobczyk - Clifford Algebra to Geometric Calculus A Unified Language for Mathematics and Physics (1984) Seite 3
\end{itemize}
\end{document}
