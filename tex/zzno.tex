\documentclass{sajzk}

\begin{document}
\section{Konjunktor-Einführungsregel \texorpdfstring{($\land$E)}{(and E)}}
\label{zzno}
\lhead{:mathe:logik:kalkül:}
Enthält eine Ableitung die Formel A als auch die Formel B, dann darfi die
Konjunktion der beiden Formeln in einer neuen Zeile aufgenommen werden. Die
Annahnmeliste der euen Zeile, wird zusammengesetzt aus den beiden
Annahnmelisten der beiden Oberformeln.

\begin{center}
  \begin{minipage}[t]{0.4\textwidth}
    \begin{tabular}{|c|c|c|c|}
      \hline
      Ann.               & Nr.    & Formel       & Regel \\
      \hline
      $\alpha$           & (k)    & A            & AE \\
      \hline
      $\beta$            & (l)    & B            &  \\
      \hline
      $\alpha, \beta$    & (m)    & A $\land$ B  & k,l $\land$E \\
      \hline
    \end{tabular}
  \end{minipage}
  \begin{minipage}[t]{0.4\textwidth}
    \begin{tabular}{|c|c|c|c|}
      \hline
      Ann.               & Nr.    & Formel       & Regel \\
      \hline
      $\alpha$           & (k)    & A            & AE \\
      \hline
      $\beta$            & (l)    & B            &  \\
      \hline
      $\alpha, \beta$    & (m)    & B $\land$ A  & l,k $\land$E \\
      \hline
    \end{tabular}
  \end{minipage}
\end{center}

\textsc{Beispiele:}
\begin{center}
    \[P, Q \therefore P \land Q\] \\
\begin{tabular}{|c|c|c|c|}
  \hline
  Annahme            & Nummer & Formel       & Regel \\
  \hline
  1                  & (1)    & $P$          & AE \\
  \hline
  2                  & (2)    & $Q$          & AE \\
  \hline
  3                  & (3)    & $P \land Q$  & 1,2 $\land$E \\
  \hline
\end{tabular}
\end{center}
Es gilt: $P, Q \vdash_J P \land Q$
\newpage
\begin{center}
    \[P \ra Q, P \ra R \therefore P \ra (Q\land R)\]
\begin{tabular}{|c|c|c|c|}
  \hline
  Annahme            & Nummer & Formel                 & Regel \\
  \hline
  1                  & (1)    & $P \ra Q$              & AE \\
  \hline
  2                  & (2)    & $P \ra R$              & AE \\
  \hline
  $3^\star$          & (3)    & $P$                    & AE \\
  \hline
  $1, 3^\star$       & (4)    & $Q$                    & 1,3 MPP \\
  \hline
  $2, 3^\star$       & (5)    & $R$                    & 2,3 MPP \\
  \hline
  $1, 2, 3^\star$    & (6)    & $Q \land R$            & 4,5 $\land$E \\
  \hline
  $1, 2$             & (7)    & $P \ra (Q \land R)$    & 3,6 K \\
  \hline
\end{tabular}
\end{center}
Es gilt: $P \ra Q, P \ra R \vdash_J P \ra (Q\land R)$
\subsection{Referenzen}
\begin{itemize}
    \item \href{8hkr.pdf}{Kalkül der Aussagenlogik} 8hkr.tex
\end{itemize}

\subsection{Literatur}
\begin{itemize}
    \item Timm Lampert - Klassische Logik (2004) Seite 99-100
\end{itemize}
\end{document}
