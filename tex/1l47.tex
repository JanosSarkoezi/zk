\documentclass{sajzk}

\begin{document}
\section{Summe von Punkten am gleichen Ort}
\label{1l47}
\lhead{:mathe:ga:}

Sind die beiden Punkte $e_1$ und $e_2$ an der selben Stelle $f$ des Raumes,
haben aber verscheiden Massen, so können sie folgendermaßen summiert werden
$$e_1 + e_2 = \mathfrak{m}_1f + \mathfrak{m}_2f = (\mathfrak{m}_1 + \mathfrak{m}_2)f$$

\subsection{Referenzen}
\begin{itemize}
    \item \href{9wc5}{Summe von Punkten} 9wc5.tex
\end{itemize}

\subsection{Literatur}
\begin{itemize}
    \item Hermann Grassmann - Projektive Geometrie der Ebene Unter Benutzung der Punktrechnung Dargestellt Erster Band Binäres (1909) Seite 1
\end{itemize}
\end{document}
