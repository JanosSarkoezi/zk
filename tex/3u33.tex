\documentclass{sajzk}

\begin{document}
\section{Ableitungen} 
\label{3u33}
\lhead{:mathe:logik:kalkül:}

Eine Ableigung einer $K$-Formel aus $Pr$-Formeln, ist eine kette von Formel,
deren Anfangsglieder die $Pr$-Formeln sind, deren Endglied die $K$-Formel ist
und deren semtliche Übergänge gemäß Ableitungsregeln vorgenommen weden.

Im folgenden werden die Regeln dargestellt. Hierbei werden die Buchstaben
$'A', 'B', 'C'$ als \textit{Metavariablen} für die Formeln der Spache $J$
verwendet.

Tabellarische Darstellung
\begin{center}
\begin{tabular}{|c|c|c|c|}
  \hline
  Annahme & Nummer & Formel & Regel \\
  \hline
  1,2     & (4)    & Q      & 1,3 MPP \\
  \hline
\end{tabular}
\end{center}

\textsc{Paraphrase:}

Aus den Formeln, die in den $Zeilen$ $1$ $und$ $2$ stehen, ist die Formel
$'Q'$ in der $Zeile$ $(4)$ ableitbar. Die Ableitung erfogt auf Grund der Regel
MPP angewendet auf die Formeln in den $Zeilen$ $1$ $und$ $3$.
\subsection{Referenzen} 

\begin{itemize}
    \item \href{8hkr.pdf}{Kalkül der Aussagenlogik} 8hkr.tex
\end{itemize}

\subsection{Literatur} 

\begin{itemize}
  \item Timm Lampert - Klassische Logik (2004) Seite 88
\end{itemize}
\end{document}
