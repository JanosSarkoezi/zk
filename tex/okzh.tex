\documentclass{sajzk}

\usepackage{minted}

\begin{document}
\section{Systemd Unit}
\label{okzh}
\lhead{:systemd:user:unit:}

Eine Unit-Konfigurationsdatei ist eine reine Textdatei im Ini-Format, die
Informationen über einen Dienst, ein Socket, ein Gerät, einen Einhängepunkt,
einen Selbsteinhängepunkt, eine Auslagerungsdatei oder -partition, ein
Startziel, einen überwachten Dateipfad, einen von systemd(1) gesteuerten und
überwachten Zeitgeber, eine Ressourcenverwaltungsscheibe oder eine Gruppe von
extern erstellten Prozessen kodiert.

\textsc{Beispiele:}
Hier soll als Beispiel eine zeitgesteuerte Nachricht erscheinen. Diese
Nachricht informiert den Benutzer, dass er Wasser trinken soll. Dafür werden
drei Dateien erstellt. Zuerst eine Shell Datei namens wasser.sh. Die Shell
Datei muss ausführbar gemacht werden (chmod u+x wasser.sh)

\begin{minted}{bash}
#!/bin/bash
export DBUS_SESSION_BUS_ADDRESS = \
"${DBUS_SESSION_BUS_ADDRESS:-unix:path=/run/user/${UID}/bus}"

/usr/bin/notify-send 'Wasser Trinken' \
"Du solltest etwas wasser trinken $(date --iso-8601=seconds)." \
-i dialog-information
\end{minted}
\newpage
Dann ein Service-Datei mit dem Namen wasser.service
\begin{minted}{ini}
[Unit]
Description = Wasser trinken

[Service]
Type      = oneshot
ExecStart = /home/<user>/.config/systemd/user/wasser.sh

[Install]
WantedBy = multi-user.target
\end{minted}
\textsc{Bemerkung:} %Die Zeichen \lstinline!<user>! müssen durch den
Benutzernamen ersetzt werden.
\newpage
Zu letzt die Timer-Datei wasser.timer
%\begin{mdframed}[backgroundcolor=light-gray, roundcorner=10pt,leftmargin=1, rightmargin=1, innerleftmargin=15, innertopmargin=15,innerbottommargin=15, outerlinewidth=1, linecolor=light-gray]
%\begin{lstlisting}
\begin{minted}{ini}
[Unit]
Description = Timer Wasser trinken

[Timer]
OnCalendar  = *:0/20
AccuracySec = 100ms
Unit        = wasser.service

[Install]
WantedBy = multi-user.target
\end{minted}
%\end{lstlisting}
%\end{mdframed}

\subsection{Referenzen}
\begin{itemize}
    \item \href{gjxh.pdf}{Systemd} gjxh.tex
\end{itemize}

\subsection{Literatur}
\begin{itemize}
    \item man:systemd.unit(5)
\end{itemize}
\end{document}
