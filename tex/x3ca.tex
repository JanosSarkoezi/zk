\documentclass{sajzk}

\begin{document}
\section{Das äußere Produkt zwischen zwei Vektoren} 
\label{x3ca}
\lhead{:mathe:ga:}

Wird die Differenz der Gleichungen
\begin{align*}
  ab &= a\cdot b + a\wedge b \\ 
  ba &= b\cdot a + b\wedge a
\end{align*}
gebildet und unter Benutzung der antikommutativität $a\wedge b = -b\wedge a$ des äußeren
Produts und der kommutativität $a\cdot b = b\cdot a$ des Skalarprodukts
zwischen zwei Vektoren, erhalten wir die
Gleichung
$$ a\wedge b = \frac{1}{2}(ab-ba)$$
Das heißt, das äußere Produkt kann mit dem geometrischen Produkt dargestellt
werden.

\subsection{Referenzen} 
\begin{itemize}
    \item \href{f35d.pdf}{Geometrische Algebra} f35d.tex
    \item \href{bzmt.pdf}{Das innere und äußere Produkt} bzmt.tex
\end{itemize}

\subsection{Literatur}
\begin{itemize}
    \item Alan Macdonald - Linear and Geometric Algebra (2021) Seite 83
    \item David Hestenes - New Foundations for Classical Mechanics (2002) Seite 39
\end{itemize}
\end{document}
