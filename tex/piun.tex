\documentclass{sajzk}

\begin{document}
\section{\texorpdfstring{$k$-Blades nach Hestenes}{k-Blades nach Hestenes}}
\label{piun}
\lhead{:mathe:ga:}

In seinem Buch über die klassische Mechanik wird das $k$-Blade etwas schwammig
eingeführt. Da vermischt er die Begriffe $k$-Vektor mit dem $k$-Blade.

In dem Buch über Clifford Algebra sieht es schon viel besser aus. Dort steht

\begin{quote}
Ein Multivektor $A_r$ wird als $r$-Blade oder einfacher $r$-Vektor
bezeichnet, genau dann, wenn er in $r$ antikommutierende Vektoren
$a_1,...,a_r$ **faktorisiert** werden kann. Das heißt

$$
A_r=a_1...a_r
$$

wobei

$$
a_ja_k=-a_ka_j
$$

für $j,k=1,...,r$ und $k\neq j$ gilt.
\end{quote}

\subsection{Referenzen} 
\begin{itemize}
    \item \href{f35d.pdf}{Geometrische Algebra} f35d.tex
    \item \href{yjez.pdf}{Linearkombination} yjez.tex
    \item \href{zju4.pdf}{Grahm-Schmidtsches Orthogonalisieung} zju4.tex
    \item \href{93t3.pdf}{$k$-Vektoren} 93t3.tex
    \item \href{kikd.pdf}{$k$-Blades} kikd.tex
\end{itemize}

\subsection{Literatur}
\begin{itemize}
    \item David Hestenes - New Foundations for Classical Mechanics (2002) Seite 34
    \item David Hestenes, Garret Sobczyk - Clifford Algebra to Geometric Calculus A
          Unified Language for Mathematics and Physics (1984) Seite 4
\end{itemize}
\end{document}
