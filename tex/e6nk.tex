\documentclass{sajzk}

\usepackage{mathtools}

\begin{document}
\section{Basisvektoren für GA in 2D}
\label{e6nk}
\lhead{:mathe:ga:}

Sind die Vektoren $e_1$ und $e_2$ die Standardbasisvektoren des
zweidimensionalen Raumens $\R^2$ (sind orthonormiert). So gibt es vier
Basisvektoren des $\mathscr{G}^2$.

\[
\begin{matrix}
    1 \\ 
    e_1\hspace{3ex} e_2 \\ 
    e_1e_2
\end{matrix}
\hspace{3ex}
\begin{matrix*}[l]
    \textrm{basis für }0\textrm{-Vektoren (Skalare)} \\ 
    \textrm{basis für }1\textrm{-Vektoren (Vektoren)} \\ 
    \textrm{basis für }2\textrm{-Vektoren (Bivektoren)} \\ 
\end{matrix*}
\]

Mit $e_1e_2$ wird das geometische Produkt der bieden Vektoren $e_1$ und $e_2$
gemeint.

\textsc{Bemerkung:} Da die beiden Vektoren $e_1$ und $e_2$ senkrecht auf ein ander
stehen, ist $e_1e_2 = e_1\wedge e_2$.

\subsection{Referenzen}
\begin{itemize}
    \item \href{f35d.pdf}{Geometrische Algebra} f35d.tex
    \item \href{81js.pdf}{Geometrisches Produkt} 81js.tex
    \item \href{kikd.pdf}{$k$-Blades} kikd.tex
    \item \href{93t3.pdf}{$k$-Vektoren} 93t3.tex
\end{itemize}

- [Ein Multivektor](d1fv.md)

\subsection{Literatur}
\begin{itemize}
    \item Alan Macdonald - Linear and Geometric Algebra (2021) Seite 76
\end{itemize}
\end{document}
