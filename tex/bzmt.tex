\documentclass{sajzk}

\begin{document}
\section{Das innere und äußere Produkt}
\label{bzmt}
\lhead{:mathe:ga:}
Das innere Produkt zwischen zwei Multivektoren kann in mehreren Schritten
aufgebaut werden.
\begin{itemize}
    \item \href{1l7s.pdf}{Das innere Produkt zwischen zwei Vektoren} 1l7s.tex
    \item \href{ozr9.pdf}{Das innere Produkt zwischen einem Vektor und ein Blade} ozr9.tex
    \item \href{.pdf}{Das innere Produkt zwischen zwei Blades} .tex
    \item \href{.pdf}{Das innere Produkt zwischen einem Blade und ein Multivektor} .tex
    \item \href{.pdf}{Das innere Produkt zwischen zwei Multivektoren} .tex
\end{itemize}

Die selbe Überlegung kann auch auf das äußere Produkt angewandt werden:
\begin{itemize}
    \item \href{x3ca.pdf}{Das äußere Produkt zwischen zwei Vektoren} x3ca.tex
    \item \href{.pdf}{Das äußere Produkt zwischen einem Vektor und ein Blade} .tex
    \item \href{.pdf}{Das äußere Produkt zwischen zwei Blades} .tex
    \item \href{.pdf}{Das äußere Produkt zwischen einem Blade und ein Multivektor} .tex
    \item \href{.pdf}{Das äußere Produkt zwischen zwei Multivektoren} .tex
\end{itemize}

% Das innere Produkt wird mit zwei Multivektoren gebildet. Am besten erklärt das
% Hestenes. Die Bildung des inneren Prokutes wird in zwei Schritten erklärt.
% \begin{itemize}
%   \item Das innere Produkt zwischen homogene Multiveltoren $A_r$ und $B_s$ ist durch
%   $$A_r\cdot B_s = \langle A_rB_s\rangle_{|r-s|}$$
%   wenn $r, s > 0$ ist und
%   $$A_r\cdot B_s = 0$$
%   wenn $r = 0$ oder $s = 0$ ist.
%   \item Seien zwei beliebige Multiveltoren $A$ und $B$ gegeben. Das innere
%   Produkt von $A$ mit $B$ wird als
%   $$A\cdot B = \sum_r \langle A\rangle_r\cdot B = \sum_s A \cdot \langle
%   B\rangle_s = \sum_r\sum_s\langle A\rangle_r \cdot \langle B\rangle_s$$
% \end{itemize}
% 
% ## Das innere Produkt zwischen Multivektoren
% \begin{itemize}
% - [Das innere Produkt von Macdonald](9erq.md)
% - [Das innere Produkt von Hestenes](c0kd.md)
% \begin{itemize}
% 
% ## Referenzen
% - [Geometrische Algebra](f35d.md)
% - [Reeller Vektorraum](3g4f.md)
% - [Geometrisches Produkt](81js.md)
% - [Projektion und Rejektion](fmjb.md)
% - [Ein Multivektor](d1fv.md)
% - [Der $k$-Vektorteil eines Multivektors](oagu.md)
% - [Die Norm eines Multivektors](deuk.md)
% 
% :mathe:ga:
% 
% ## Literatur
% - Alan Macdonald - Linear and Geometric Algebra (2021) Seite 51
% - David Hestenes, Garret Sobczyk - Clifford Algebra to Geometric Calculus A
%   Unified Language for Mathematics and Physics (1984) Seite 6
% - David Hestenes - New Foundations for Classical Mechanics (2002) Seite 43
\end{document}
