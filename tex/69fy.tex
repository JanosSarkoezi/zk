\documentclass{sajzk}

\begin{document}
\section{Modus Ponendo Ponens (MPP)}
\label{69fy}
\lhead{:mathe:logik:kalkül:}

Enthält eine Ableitung sowohl eine Formel mit $'\ra'$ als Hauptjunktor als auch
das Antezedenz dieser Formel, dann darf ihr Konsequenz in eine \textit{neue}
Zeile aufgenommen werden. Die Annahmeliste der neuen Zeile wird zusammengesetzt
aus den Annahmelisten der beiden Oberformeln.

\begin{center}
\begin{tabular}{|c|c|c|c|}
  \hline
  Annahme         & Nummer & Formel        & Regel \\
  \hline
  $\alpha$        & (k)    & A             &  \\
  \hline
  $\beta$         & (l)    & A $\ra$ B     &  \\
  \hline
  $\alpha, \beta$ & (m)    & B             & l,k MPP \\
  \hline
\end{tabular}
\end{center}

Die Reihenfolge des Vorkommens der Oberformeln ist beliebig.

\textsc{Beispiele:}
\begin{center}
    \[P, P \ra Q \therefore Q\] \\

\begin{tabular}{|c|c|c|c|}
  \hline
  Annahme      & Nummer & Formel & Regel \\
  \hline
  1   & (1)    & $P$             & AE \\
  \hline
  2   & (2)    & $P \ra Q$       & AE \\
  \hline
  1,2 & (3)    & $Q$             & 2,1 MPP \\
  \hline
\end{tabular}
\end{center}

Es gilt: $P, P \ra Q \vdash_J Q$
\newpage
\begin{center}
    \[P \ra (Q\ra R), P \ra Q, P \therefore R\] \\

\begin{tabular}{|c|c|c|c|}
  \hline
  Annahme        & Nummer & Formel   & Regel \\
  \hline
  1     & (1)    & $P \ra (Q\ra R)$  & AE \\
  \hline
  2     & (2)    & $P \ra Q$         & AE \\
  \hline
  3     & (3)    & $P$               & AE \\
  \hline
  1,3   & (4)    & $Q \ra R$         & 1,3 MPP \\
  \hline
  2,3   & (5)    & $Q$               & 2,3 MPP \\
  \hline
  1,2,3 & (6)    & $R$               & 4,5 MPP \\
  \hline
\end{tabular}
\end{center}

Es gilt: $P \ra (Q\ra R), P \ra Q, P \vdash_J R$
\subsection{Referenzen}
\begin{itemize}
    \item \href{8hkr.pdf}{Kalkül der Aussagenlogik} 8hkr.tex
\end{itemize}

\subsection{Literatur}
\begin{itemize}
    \item Timm Lampert - Klassische Logik (2004) Seite 92-93
\end{itemize}
\end{document}
