\documentclass{sajzk}

\begin{document}
\section{Disjunktor-Beseitigungsregel \texorpdfstring{($\lor$B)}{(or B)}}
\label{4sni}
\lhead{:mathe:logik:kalkül:}
Enthält eine Ableigung eine Disjunktion der Formeln A und B, und wurde eine
beliebige Formel C einerseits aus der Annahme A, andererseits aus der Annahme B
abgeleitet, dann darf C in einer neue Zeile der Ableigung aufgenommen werden.
Die neue Annhmeliste setzt sich zusammen aus der Annhmeliste der Disjunktion A
und B und der Annhmeliste der Ableigung C aus A, vermindert um die Zeilennumer
von A, und der Annhmeliste der Ableigung C aus B, vermindert um die Zeilennumer
von B.

\begin{center}
\begin{tabular}{|c|c|c|c|}
  \hline
  Annahme                  & Nummer & Formel       & Regel \\
  \hline
  $\alpha$                 & (k)    & A $\lor$ B   & AE \\
  \hline
  $l^\star$                & (l)    & A            & AE \\
  \hline
  $\beta, l^\star$         & (m)    & C            & \\
  \hline
  $n^\star$                & (n)    & B            & AE \\
  \hline
  $\gamma, l^\star$        & (o)    & C            & \\
  \hline
  $\alpha, \beta, \gamma$  & (p)    & C            & k;l,m;n,o $\lor$B \\
  \hline
\end{tabular}
\end{center}
\newpage
\textsc{Beispiele:}
\begin{center}
    \[P \lor Q \therefore Q \lor P\] \\
\begin{tabular}{|c|c|c|c|}
  \hline
  Annahme            & Nummer & Formel       & Regel \\
  \hline
  1                  & (1)    & $P \lor Q$   & AE \\
  \hline
  $2^\star$          & (2)    & $P$          & AE \\
  \hline
  $2^\star$          & (3)    & $Q \lor P$    & 2 $\lor$E \\
  \hline
  $4^\star$          & (4)    & $Q$          & AE \\
  \hline
  $4^\star$          & (5)    & $Q \lor P$    & 4 $\lor$E \\
  \hline
  1                  & (6)    & $Q \lor P$   & 1;2,3;4,5 $\lor$B \\
  \hline
\end{tabular}
\end{center}
Es gilt: $P \lor Q \vdash_J Q \lor P$
\begin{center}
    \[P \lor Q, P \ra R, Q \ra R \therefore R\] \\
\begin{tabular}{|c|c|c|c|}
  \hline
  Annahme            & Nummer & Formel       & Regel \\
  \hline
  1                  & (1)    & $P \lor Q$   & AE \\
  \hline
  2                  & (2)    & $P \ra R$    & AE \\
  \hline
  3                  & (3)    & $Q \ra R$    & AE \\
  \hline
  $4^\star$          & (4)    & $P$          & AE \\
  \hline
  $2, 4^\star$       & (5)    & $R$          & 2,4 MPP \\
  \hline
  $6^\star$          & (6)    & $Q$          & AE \\
  \hline
  $3, 6^\star$       & (7)    & $R$          & 3,6 MPP \\
  \hline
  1,2,3              & (5)    & $R$          & 1;4,5;6,7 $\lor$B \\
  \hline
\end{tabular}
\end{center}
Es gilt: $P \lor Q, P \ra R, Q \ra R \vdash_J R$
\subsection{Referenzen}
\begin{itemize}
    \item \href{8hkr.pdf}{Kalkül der Aussagenlogik} 8hkr.tex
\end{itemize}

\subsection{Literatur}
\begin{itemize}
    \item Timm Lampert - Klassische Logik (2004) Seite 101-102
\end{itemize}
\end{document}
