\documentclass{sajzk}

\begin{document}
\section{Annahmen Einführung (AE)}
\label{yyix}
\lhead{:mathe:logik:kalkül:}

Eine beliebige Formel A von J darf in einer beliebigen Zeile k als
\textit{Annahme} eingeführt werden. Die Annahmeliste enthält die und nur die
Zeile, in der die Annahme eingeführt wird (=k)

\begin{center}
\begin{tabular}{|c|c|c|c|}
  \hline
  Annahme & Nummer & Formel & Regel \\
  \hline
  k     & (k)    & A      & AE \\
  \hline
\end{tabular}
\end{center}

Mit dieser Regel lässt sich das Argument $P \therefore P$ Ableiten.
\begin{center}
\begin{tabular}{|c|c|c|c|}
  \hline
  Annahme & Nummer & Formel & Regel \\
  \hline
  1     & (1)    & $P$      & AE \\
  \hline
\end{tabular}
\end{center}

\textsc{Paraphrase:}

Aus der Formel, die in Zeile 1 steht ist die Formel $'P'$ in der Zeile 1
ableitbar. Die Ableitung erfolgt auf Grund der Regel AE.

Es gilt: $P \vdash_J P$


\subsection{Referenzen}
\begin{itemize}
  \item \href{8hkr.pdf}{Kalkül der Aussagenlogik} 8hkr.tex
\end{itemize}

\subsection{Literatur}
\begin{itemize}
    \item Timm Lampert - Klassische Logik (2004) Seite 91-92
\end{itemize}
\end{document}
