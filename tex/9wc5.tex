\documentclass{sajzk}

\begin{document}
\section{Summe von Punkten}
\label{9wc5}
\lhead{:mathe:ga:}

Für die Summe von Punkten im Raum wird ein Punkt mit zwei Daten karakterisiert
\begin{itemize}
    \item Ein Zahlenfaktor $\mathfrak{m}$, genannt die Masse des Punktes
    \item Die Lage $f$ des Punktes im Raum
\end{itemize}

Somit kann ein Punkt $e$ als
$$
e = \mathfrak{m}f
$$
geschrieben werden.

\subsection{Referenzen}
\begin{itemize}
    \item \href{16ea.pdf}{Grassmanns Idee} 16ea.tex
    \item \href{1l47.pdf}{Summe von Punkten am gleichen Ort} 1l47.tex
    \item \href{9gpd.pdf}{Summe von Punkten an verscheiden Orten} 9gpd.tex
\end{itemize}

\subsection{Literatur}
\begin{itemize}
    \item Hermann Grassmann - Projektive Geometrie der Ebene Unter Benutzung der Punktrechnung Dargestellt Erster Band Binäres (1909) Seite 1
\end{itemize}
\end{document}
