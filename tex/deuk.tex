\documentclass{sajzk}

\begin{document}
\section{Die Norm eines Multivektors}
\label{deuk}
\lhead{:mathe:ga:}
Die Norm eines Multivektors $A$ wird als
\[|A|=\langle A^\dagger A \rangle_0^{1/2}\]
definiert.

\textsc{Bemerkungen:}
\begin{itemize}
  \item Aus $\langle A^\dagger A \rangle_0$ kann die Wurzel gezogen werden, da
    \[
    \langle A^\dagger A \rangle_0 \geq 0
    \]
\item Die Norm eines Multivektors $A$ der Form $A = a_1\cdots a_k$ ist
  \[
  |A|^2 = \langle (a_1\cdots a_k)^\dagger(a_1\cdots a_k) \rangle_0 = |a_1|^2\cdots |a_k|^2
  \]
\end{itemize}

\subsection{Referenzen}
\begin{itemize}
    \item \href{f35d.pdf}{Geometrische Algebra} f35d.tex
    \item \href{d1fv.pdf}{Ein Multivektor} d1fv.tex
    \item \href{eelx.pdf}{Die Umkehrung eines Multivektors} eelx.tex
\end{itemize}

\subsection{Literatur}
\begin{itemize}
  \item Alan Macdonald - Linear and Geometric Algebra (2021) Seite 82
  \item David Hestenes - New Foundations for Classical Mechanics (2002) Seite 46
\end{itemize}
\end{document}
