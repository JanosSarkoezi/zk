\documentclass{sajzk}

\begin{document}
\section{\texorpdfstring{$k$-Blades}{k-Blades}}
\label{kikd}
\lhead{:mathe:ga:}
$k$-Blades können auf verschiedene Weise definiert werden.

\begin{itemize}
    \item \href{9o91.pdf}{$k$-Blades nach Macdonalds} 9o91.tex
    \item \href{piun.pdf}{$k$-Blades nach Hestenes} piun.tex
    \item \href{iqlf.pdf}{$k$-Blades nach Chisolm} iqlf.tex
\end{itemize}

\subsection{Referenzen}
\begin{itemize}
    \item \href{f35d.pdf}{Geometrische Algebra} f35d.tex
    \item \href{yjez.pdf}{Linearkombination} yjez.tex
    \item \href{zju4.pdf}{Grahm-Schmidtsches Orthogonalisieung} zju4.tex
    \item \href{93t3.pdf}{$k$-Vektoren} 93t3.tex
\end{itemize}

\subsection{Literatur}
\begin{itemize}
    \item Alan Macdonald - Linear and Geometric Algebra (2021) Seite 92
    \item Alan Macdonald - Linear and Geometric Algebra (2021) Seite 103
    \item David Hestenes - New Foundations for Classical Mechanics (2002) Seite 34
    \item David Hestenes, Garret Sobczyk - Clifford Algebra to Geometric Calculus A
          Unified Language for Mathematics and Physics (1984) Seite 4
    \item Eric Chisolm - Geomeric Algebra (2012) Seite 12
\end{itemize}
\end{document}
