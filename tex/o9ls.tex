\documentclass{sajzk}

\begin{document}
\section{Summe von Punkten an verscheiden Orten}
\section{Reductio Ad Absurdum (RAA)}
\label{o9ls}
\lhead{:mathe:logik:kalkül:}

Enthält eine Ableitung eine $FK$-Formel, und hängt dieser unter anderem von der
Annahme A ab, dann darf die Negation von A in deine neue Zeile aufgenommen
werden. Die neue Annahmeliste ist eine Kopie der Annahmeliste der $FK$-Formel
verkürzt um die Zeilennummer der Annahme A.

\begin{center}
\begin{tabular}{|c|c|c|c|}
  \hline
  Annahme                  & Nummer & Formel            & Regel \\
  \hline
  $k^\star$                & (k)    & A                 & AE \\
  \hline
  $\alpha, k^\star$        & (l)    & B $\land\lnot$B   & AE \\
  \hline
  $\alpha$                 & (m)    & $\lnot$A          & k,l RAA\\
  \hline
\end{tabular}
\end{center}
\textsc{Bemerkung:} FK steht für formale Kontradiktion, also für $P\land\lnot
P$.

\textsc{Beispiele:}
\begin{center}
    \[P \ra Q, P \ra\lnot Q \therefore \lnot Q \] \\
\begin{tabular}{|c|c|c|c|}
  \hline
  Annahme            & Nummer & Formel           & Regel \\
  \hline
  1                  & (1)    & $P \ra Q$        & AE \\
  \hline
  2                  & (2)    & $P \ra\lnot Q$   & AE \\
  \hline
  $3^\star$          & (3)    & $P$              & AE \\
  \hline
  $1, 3^\star$       & (4)    & $Q$              & 1,3 MPP \\
  \hline
  $2, 3^\star$       & (5)    & $\lnot Q$        & 2,3 MPP \\
  \hline
  $1,2,3^\star$      & (6)    & $Q \land\lnot Q$ & 4,5 $\land$E \\
  \hline
  1,2                & (7)    & $\lnot P$        & 3,6 RAA \\
  \hline
\end{tabular}
\end{center}
Es gilt: $P \lor Q \vdash_J Q \lor P$
\subsection{Referenzen}
\begin{itemize}
    \item \href{8hkr.pdf}{Kalkül der Aussagenlogik} 8hkr.tex
\end{itemize}

\subsection{Literatur}
\begin{itemize}
    \item Timm Lampert - Klassische Logik (2004) Seite 101-102
\end{itemize}
\end{document}
