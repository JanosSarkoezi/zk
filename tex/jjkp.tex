\documentclass{sajzk}

\begin{document}
\section{Gesetz vom ausgeschlossenen Dritten (Tertium non Datur)(GTD)}
\label{jjkp}
\lhead{:mathe:logik:kalkül:}
\textbf{GTD:} $\vdash A\lor\lnot A$

\textbf{Theorem} $\vdash P\lor\lnot P$

\textbf{Ableitung:}
\begin{center}
\begin{tabular}{|c|c|c|c|}
  \hline
  Annahme            & Nr. & Formel                                       & Regel \\
  \hline
  $1^\star$          & (1)    & $\lnot(P\lor\lnot P)$                     & AE \\
  \hline
  $2^\star$          & (2)    & $P$                                       & AE \\
  \hline
  $2^\star$          & (3)    & $P\lor\lnot P$                            & 2 $\lor$E \\
  \hline
  $1^\star,2^\star$  & (4)    & $(P\lor\lnot P)\land\lnot(P\lor\lnot P)$  & 3,1 $\land$E \\
  \hline
  $1^\star$          & (5)    & $\lnot P$                                 & 2,4 RAA \\
  \hline
  $1^\star$          & (6)    & $P\lor\lnot P$                            & 5 $\lor$E \\
  \hline
  $1^\star$          & (7)    & $(P\lor\lnot P)\land\lnot(P\lor\lnot P)$  & 6,1 $\land$E \\
  \hline
                     & (8)    & $\lnot\lnot(P\lor\lnot P)$                & 1,7 RAA \\
  \hline
                     & (9)    & $P\lor\lnot P$                            & 7 DNB \\
  \hline
\end{tabular}
\end{center}

\subsection{Referenzen}
\begin{itemize}
    \item \href{s52i.pdf}{Logische Gesetze} s52i.tex
\end{itemize}

\subsection{Literatur}
\begin{itemize}
    \item Timm Lampert - Klassische Logik (2004) Seite 121
\end{itemize}
\end{document}
