\documentclass{sajzk}

\begin{document}
\section{Doppelte Negation (DNE, DNB)}
\label{is2y}
\lhead{:mathe:logik:kalkül:}
\textsc{DNE}: Enthält eine Ableitung eine Formel A, dann darf ihre doppelte
Negation in eine neue Zeile aufgenommen werden.

\textsc{DNB}: Enthält eine Ableitung die doppelte Negation einer Formel A, dann
darf A in eine neue Zeile aufgenommen werden.

Die neue Annahmeliste ist jeweils eine Kopie der Annahmeliste der Oberformel.

\begin{center}
  \begin{minipage}[t]{0.4\textwidth}
    \begin{tabular}{|c|c|c|c|}
      \hline
      Ann.               & Nr.    & Formel         & Regel \\
      \hline
      $\alpha$           & (k)    & A              & AE \\
      \hline
      $\alpha$           & (l)    & $\lnot\lnot$ A & k DNE \\
      \hline
    \end{tabular}
  \end{minipage}
  \begin{minipage}[t]{0.4\textwidth}
    \begin{tabular}{|c|c|c|c|}
      \hline
      Ann.               & Nr.    & Formel       & Regel \\
      \hline
      $\alpha$           & (k)    & $\lnot\lnot$ A & AE \\
      \hline
      $\alpha$           & (l)    & A              & k DNB \\
      \hline
    \end{tabular}
  \end{minipage}
\end{center}
\textsc{Beispiele:}
\begin{center}
    \[P \therefore \lnot\lnot P \] \\
\begin{tabular}{|c|c|c|c|}
  \hline
  Annahme            & Nummer & Formel         & Regel \\
  \hline
  1                  & (1)    & $P$            & AE \\
  \hline
  2                  & (2)    & $\lnot\lnot P$ & 1 DNE \\
  \hline
\end{tabular}
\end{center}
Es gilt: $P \vdash_J \lnot\lnot P$
\newpage
\begin{center}
    \[P \land\lnot P\therefore Q \] \\
\begin{tabular}{|c|c|c|c|}
  \hline
  Annahme            & Nummer & Formel                          & Regel \\
  \hline
  1                  & (1)    & $P\land\lnot P$                 & AE \\
  \hline
  $2^\star$          & (2)    & $\lnot Q$                       & 1 DNE \\
  \hline
  $1,2^\star$        & (3)    & $(P\land\lnot\ P)\land \lnot Q$ & 1,2 $\land$E \\
  \hline
  $1,2^\star$        & (4)    & $P\land\lnot P$           & 3 $\land$B \\
  \hline
  1                  & (5)    & $\lnot\lnot Q$            & 2,4 RAA \\
  \hline
  1                  & (6)    & $Q$                       & 5 DNB \\
  \hline
\end{tabular}
\end{center}
Es gilt: $P \land\lnot P \vdash_J Q$
\begin{center}
    \[P \ra\lnot Q, Q\therefore \lnot P \] \\
\begin{tabular}{|c|c|c|c|}
  \hline
  Annahme            & Nummer & Formel                    & Regel \\
  \hline
  1                  & (1)    & $P\ra\lnot Q$             & AE \\
  \hline
  2                  & (2)    & $Q$                       & AE \\
  \hline
  2                  & (3)    & $\lnot\lnot Q$            & 2 DNE \\
  \hline
  1,2                & (4)    & $\lnot P$                 & 1,3 MTT \\
  \hline
\end{tabular}
\end{center}
Es gilt: $P \ra\lnot Q, Q \vdash_J \lnot P$
\subsection{Referenzen}
\begin{itemize}
    \item \href{8hkr.pdf}{Kalkül der Aussagenlogik} 8hkr.tex
\end{itemize}

\subsection{Literatur}
\begin{itemize}
    \item Timm Lampert - Klassische Logik (2004) Seite 108-109
\end{itemize}
\end{document}
