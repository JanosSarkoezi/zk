\documentclass{sajzk}

\begin{document}
\section{Logische Gesetze}
\label{s52i}
\lhead{:mathe:logik:kalkül:}
Logische Gesetze sind Metaformeln, deren Einsetzungsinstanzen Theoreme sind.

Theoreme sind Formeln, die mit leerer Annahmeliste ableitbar sind.

Kann eine Instanz einer Metaformel als Theorem abgeleitet werden, dann können
alle Instanzen einer Metaformel als Theorem abgeleitet werden.

\subsection{Referenzen}
\begin{itemize}
    \item \href{3z0c.pdf}{Gesetz vom ausgeschlossenen Widerspruch (GAW)} 3z0c.tex
    \item \href{jjkp.pdf}{Gesetz vom ausgeschlossenen Dritten (Tertium non Datur)(GTD)} jjkp.tex
    \item \href{hubq.pdf}{Gesetz der doppelten Negation (GDN)} hubq.tex
\end{itemize}

\subsection{Literatur}
\begin{itemize}
    \item Timm Lampert - Klassische Logik (2004) Seite 119-120
\end{itemize}
\end{document}
