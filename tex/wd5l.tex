\documentclass{sajzk}

\begin{document}
\section{Schlussregeln}
\label{wd5l}
\lhead{:mathe:logik:kalkül:}
Schlussregeln sind Metaformeln, die aussagen, dass $K$-Formeln einer bestimmten
Form aus $Pr$-Formeln einer bestimmten Form ableitbar sind.

\subsection{Referenzen}
\begin{itemize}
    \item \href{3rqj.pdf}{Direkte Schlussregeln} 3rqj.tex
    \item \href{0ku1.pdf}{Umformungsregeln} 0ku1.tex
    \item \href{t3a1.pdf}{Indirekte Schlussregeln} t3a1.tex
\end{itemize}

\subsection{Literatur}
\begin{itemize}
    \item Timm Lampert - Klassische Logik (2004) Seite 123
\end{itemize}
\end{document}
