\documentclass{sajzk}

\begin{document}
\section{Lineare Unabhängigkeit}
\label{v332}
\lhead{:mathe:la:}
Vektoren $v_1, ..., v_n$ sind linear Unabhängig, genau dann, wenn nur die
triviale Linearkombination die Gleichung

\[
\alpha_1 v_1 + \dots + \alpha_n v_n = 0
\]

erfüllt. Das heißt, alle $\alpha_i = 0$ für $i = 1, ..., n$.

Linear Abhängig sind die Vektoren, wenn sie nicht linear Unabhängig sind. Also
wenn es auch eine Linearkombination der Vektoren $v_1, ..., v_n$ gibt, wo nicht
alle $\alpha_i = 0$ sind. Anders gesagt, es gibt eine nicht triviale
Linearkombination der Vektoren $v_1, ..., v_n$.

\textsc{Beispiel:} Die Vektoren

\[
\begin{pmatrix}
 1 \\
 2
\end{pmatrix} ,
\begin{pmatrix}
 3 \\
 4
\end{pmatrix}
\]

erfüllen die Gleichung

\[
\alpha_1
\begin{pmatrix}
 1 \\
 2
\end{pmatrix}
+
\alpha_2
\begin{pmatrix}
 3 \\
 4
\end{pmatrix} = 0
\]

nur mit $\alpha_1 = \alpha_2 = 0$. Also es gilt nur die triviale Linearkombination.

\subsection{Referenzen}
\begin{itemize}
    \item \href{jtfm.pdf}{Lineare Algebra} jtfm.tex
    \item \href{3g4f.pdf}{Reeller Vektorraum} 3g4f.tex
    \item \href{yjez.pdf}{Linearkombination} yjez.tex
\end{itemize}

\subsection{Literatur}
\begin{itemize}
    \item Alan Macdonald - Linear and Geometric Algebra (2021) Seite 24
\end{itemize}
\end{document}
