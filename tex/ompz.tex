\documentclass{sajzk}

\begin{document}
\section{Axiome von Chisolm}
\label{ompz}
\lhead{:mathe:ga:}

Die Geometrische Algebra ist eine Menge $\mathscr{G}$ mit zwei Verknüpungen.
Eine Addition und eine Multiplikation (auch als geometisches Produkt genannt)
und gehorcht den folgenden Axiomen:
\begin{itemize}
    \item $\mathscr{G}$ ist ein Ring mit Einheit. Die additive Einheit ist die $0$ und
          die multiplikative Einheit ist die $1$.
    \item $\mathscr{G}$ enthält einen Körper $\mathscr{G}_0$ mit Charakteristik Null.
          Er enthält die $0$ und die $1$.
    \item $\mathscr{G}$ enthält eine Teilmenge $\mathscr{G}_1$, die abgeschlossen unter
          der Addition ist und wenn $\lambda\in\mathscr{G}_0$ und $v\in\mathscr{G}_1$,
          dann ist $\lambda v=v\lambda\in\mathscr{G}_1$.
    \begin{itemize}
        \item Die Elemente von $\mathscr{G}_1$ werden $1$-Vektor, homogener Vektor des
              Grades $1$ oder einfach Vektor genannt.
    \end{itemize}
\item Das Quadrat von jedem Vektor ist ein Skalar.
\item Das Skalarprodukt ist nicht degeneriert.
    \begin{itemize}
        \item Die nicht Degeneriertheit bedeutet, das nur der Nullvektor $0$ zu allen
              Vektoren senkrecht seht.
        \item Das Skalarprodukt kann mit dem geometischen Produkt definiert werden:
              $$u\cdot v :=\frac{1}{2}(uv+vu)$$
    \end{itemize}
\item Wenn $\mathscr{G}_0=\mathscr{G}_1$, dann ist $\mathscr{G}=\mathscr{G}_0$.
      Sonst ist $\mathscr{G}$ die direkte Summe von $\mathscr{G}_r$.
    \begin{itemize}
        \item Die Menge $\mathscr{G}_r$ ist die Menge der $r$-Vektoren.
        \item Ein $r$-Vektor ist die Summen von $r$-Blades.
        \item Ein $r$-Blade ist das Produkt von $r$ orthogonalen (anti kommutierenden)
              Vektoren.
        \item Wenn $\lambda\in\mathscr{G}_0$ und $A\in\mathscr{G}_r$,
              dann ist
          $$\lambda A=A\lambda\in\mathscr{G}_r$$
        \item Diese Axiom sagt aus, Wenn $A\in\mathscr{G}$, dann kann $A$ nur auf genau
              eine Art in $r$-Vektoren zerlegt werden. $A=\sum_r A_r$ mit
              $A_r\in\mathscr{G}_r$.
        \item Führt man ein Operator $\langle \rangle_r:\mathscr{G}\to\mathscr{G}_r$ ein,
              so soll dieser Operator die folgenden Aussagen erfüllen:
        \begin{itemize}
            \item $A$ ist genau dann ein $r$-Vektor, wenn $A = \langle A\rangle_r$ ist.
            \item $\langle\lambda A\rangle_r=\langle A\lambda\rangle_r=\lambda\langle A\rangle_r$.
            \item $\langle\langle A\rangle_r\rangle_s = \langle A\rangle_r\delta_r^s$
            \item $\sum_r A_r = A$ für alle $A\in\mathscr{G}$.
            \item $\langle A\rangle_r = 0$ wenn $r<0$ für alle $A\in\mathscr{G}$.
        \end{itemize}
    \end{itemize}
\end{itemize}

\subsection{Referencen}
\begin{itemize}
    \item \href{f35d.pdf}{Geometrische Algebra} f35d.tex
    \item \href{il6v.pdf}{Axiomatische sichtweise} il6v.tex
\end{itemize}

\subsection{Literatur}
\begin{itemize}
    \item Eric Chisolm - Geomeric Algebra (2012) Seite 10-12
\end{itemize}
\end{document}
