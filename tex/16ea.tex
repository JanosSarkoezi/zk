\documentclass{sajzk}

\begin{document}
\section{Grassmanns Idee}
\label{16ea}
\lhead{:mathe:ga:}

Hermann Grassmann hatte die Idee eine neue Operation, das äußere Produkt, für die Objekte

\begin{itemize}
    \item Punkte
    \item Strecken
    \item Stäbe
    \item Felder
    \item Blätter
\end{itemize}

einzuführen. Um dies zu erreichen mussten die Operationen Summation und
Differenz Bildung von diesen Objekten betrachtet werden.

\subsection{Referenzen}
\begin{itemize}
    \item \href{9wc5.pdf}{Summe von Punkten} 9wc5.tex
\end{itemize}

\subsection{Literatur}
\begin{itemize}
    \item Hermann Grassmann - Projektive Geometrie der Ebene Unter Benutzung der Punktrechnung Dargestellt Erster Band Binäres (1909)
\end{itemize}
\end{document}
