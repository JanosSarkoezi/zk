\documentclass{sajzk}

\begin{document}
\section{\texorpdfstring{\texorpdfstring{$k$-Blades nach Macdonalds}{k-Blades nach Macdonalds}}{k-Blades}}
\label{9o91}
\lhead{:mathe:ga:}
Ein $k$-Blade $B\space(k>0)$ des geometischen Raumes $\mathscr{G}^n$ ist das
geometische Produkt von $k$ paarweise orthogonalen Vektoren.

\[
B=b_1\cdots b_k
\]

ein $0$-Blade ist eine skalare Zalh ungleich null.

\textsc{Bemerkungen:}
\begin{itemize}
    \item Der Maximalwert der Zahl $k$ ist gleich $n$ in $\mathscr{G}^n$.
    \item Gibt es $k$ linear Unabhängigne Vektoren $u_1, ..., u_k$ so kann ein Blade $B$
          auch geschriben werden als

  \[
  B = u_1 \wedge \cdots \wedge u_k
  \]

  das heißt

\[
u_1 \wedge \cdots \wedge u_k = b_1 \cdots b_k
\]

  Die Umformung der linear unabhängignen Vektoren in orthogonale Vektoren kann
  durch das Grahm-Schmidtsches Orthogonalisieung erreicht werden. Zusätzlich
  gilt die Norm der Blades ändert sich nicht
  
\[
  |u_1 \wedge \cdots \wedge u_k| = |b_1| \cdots |b_k|
\]
\end{itemize}
\subsection{Referenzen}
\begin{itemize}
    \item \href{f35d.pdf}{Geometrische Algebra} f35d.tex
    \item \href{yjez.pdf}{Linearkombination} yjez.tex
    \item \href{zju4.pdf}{Grahm-Schmidtsches Orthogonalisieung} zju4.tex
    \item \href{93t3.pdf}{$k$-Vektoren} 93t3.tex
    \item \href{kikd.pdf}{$k$-Blades} kikd.tex
\end{itemize}

\subsection{Literatur}
\begin{itemize}
    \item Alan Macdonald - Linear and Geometric Algebra (2021) Seite 94
    \item Alan Macdonald - Linear and Geometric Algebra (2021) Seite 103
\end{itemize}
\end{document}
